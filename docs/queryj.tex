\documentclass[dvips]{article}
\usepackage[
pdfauthor={Jose San Leandro},
pdftitle={QueryJ},
pdfcreator={pdftex},
pdfsubject={QueryJ},
pdfkeywords={QueryJ,ACM-SL,code,Java,database,DAO}
]{hyperref}
\usepackage[spanish]{babel}
\title{QueryJ}
\author{Author Jose San Leandro}
\date{\today}
\begin{document}
\newcommand{\queryj}{\textit{QueryJ} }
\newcommand{\acronim}[1]{\textsf{#1}}
\newcommand{\tech}[1]{\texttt{#1}}
\newcommand{\company}[1]{\textsf{#1}}
\maketitle
\begin{abstract}
Objetivos y funcionalidades de QueryJ
\end{abstract}
\section{Introducci\'on}
Antes de definir \textit{qu\'e} es \queryj, es preferible describir el
marco tecnol\'ogico que lo rodea, para detallar en qu\'e modo solventa
las deficiencias del mismo.
\subsection{Capas de persistencia}
En arquitecturas multicapa, es conveniente aislar en la medida de lo
posible la l\'ogica destinada a realizar b\'usquedas o modificaciones
sobre los datos persistentes, del resto.\\
En el caso general, esos datos estar\'an almacenados en una base de
datos relacional, y las operaciones sobre los mismos se representan
mediante sentencias \acronim{SQL}.\\
Restringi\'endonos a desarrollos \tech{Java}, la capa de persistencia
puede implementarse usando distintas tecnolog\'i{}as,
\begin{enumerate}
\item
  \tech{EJB}. Es parte de \tech{J2EE}, una especificaci\'on de
  \company{Sun} para entornos con m\'aquinas virtuales \tech{Java}
  distribuidas. El caso concreto de \textit{EJB de entidad} permite
  definir c\'omo gestionar el acceso a los datos, as\'i{} como la
  l\'ogica transaccional.
\item
  \tech{JDBC}. Es un subconjunto del API \tech{J2SE} para
  facilitar la comunicaci\'on con motores de bases de datos. En
  general, la l\'ogica utiliza \'unicamente \tech{JDBC}, y es el
  \tech{driver} espec\'i{}fico el que lleva a cabo tanto el env\'i{}o
  de la solicitud como la recepci\'on de la respuesta.
\item
  \tech{JDO}. Es una especificaci\'on de \company{Sun} que define
  expl\'icitamente tanto el ciclo de vida como la transaccionalidad de
  las operaciones sobre los datos. \tech{JDO} no est\'a muy extendido
  debido principalmente a su tard\'i{}a aparici\'on y a la falta de
  implementaciones.
\end{enumerate}

En el caso de arquitecturas no distribuidas, o bien aqu\'ellas que,
siendo distribuidas, utilicen un mecanismo no basado en
\tech{clusters} de m\'aquinas virtuales, las operaciones 
sobre los datos persistentes se realizan utilizando \tech{JDBC}.\\

\subsubsection{JDBC}
Como en la mayor parte de las \tech{API}s, \acronim{JDBC} no est\'a
debidamente protegido frente a usos indebidos, siendo los m\'as
comunes
\begin{enumerate}
  \item Gesti\'on de recursos. Para realizar una consulta, se debe
  establecer una conexi\'on (en general, de red) con el motor de base
  de datos, a trav\'es de la cual se realiza la solicitud y
  posteriormente se recuperan los resultados, en su caso. Una
  gesti\'on indebida de los recursos reservados genera errores
  sintom\'aticos dif\'i{}ciles de detectar.
  \item Eficiencia de las consultas. \tech{JDBC} ofrece la posibilidad
  de sacar provecho de la precompilaci\'on de sentencias \acronim{SQL}
  por parte del motor de base de datos. Sin embargo, no hay certeza de
  que se est\'e utilizando globalmente sin examinar las mismas una a
  una.
  \item Validez de las consultas. \tech{JDBC} no ofrece ning\'un
  m\'etodo de validaci\'on de consultas previa a su ejecuci\'on. Un
  error de sintaxis, por trivial que sea, obliga a recompilar la
  aplicaci\'on en su conjunto, lo que, aparte de inconveniente,
  introduce una penalizaci\'on en tiempo de desarrollo.
\end{enumerate}
Como resultado se obtiene, en la inmensa mayor\'i{}a de las ocasiones,
aplicaciones con \tech{connection} o \tech{cursor} \tech{leaks}.
En los casos poco frecuentes en los que no se reusen las conexiones,
los \tech{leaks} impiden que \'estas se puedan cerrar salvo despu\'es
de vencimiento de un tiempo de espera (\tech{timeout}).\\
En el caso general, las conexiones estar\'an gestionadas
transparentemente por un mecanismo de \tech{cache} (\tech{connection
pool} el cual eventualmente no podr\'a proveer de este tipo de
recursos conforme el n\'umero de \'estos vaya aproxim\'andose al
tama\~no de dicho \tech{pool}.\\
Como resultado, la \'unica soluci\'on pasa por liberar dichas
conexiones, que se realiza habitualmente \textbf{reiniciando} la
aplicaci\'on.\\
Es conveniente resaltar que los principales problemas de \tech{JDBC}
son a nivel de API, no de mecanismo de comunicaci\'on. Para resolver
estos problemas, es conveniente disponer de medios que impidan
que \tech{JDBC} se utilice indebidamente.

\subsubsection{Relaci\'on del dise\~no con el modelo relacional}
\tech{Java} es un lenguaje \textit{orientado a objetos}. Sin entrar en
detalles, eso implica que una aplicaci\'on \tech{Java} llevar\'a
asociada un dise\~no de clases concreta, que refleje en mayor o menor
medida las especificaciones de la misma, a nivel conceptual.\\
Un subconjunto de ese dise\~no estar\'a directamente relacionado con
el modelo relacional utilizado como medio de almacenamiento
persistente. Esta relaci\'on puede ser trivial, o por el contrario
revestir cierta complejidad.\\
Las librer\'i{}as de \tech{O/R Mapping} tienen como objetivo facilitar
la sincronizaci\'on de ambos modelos en el segundo caso. Asimismo, son
un recubrimiento de \tech{JDBC} que tratan de evitar que se use
indebidamente. Habitualmente, utilizan un mecanismo para definir las
relaciones entre los modelos, y las consultas espec\'i{}ficas, basado
en XML, de forma que queden separados del la l\'ogica de la
aplicaci\'on. Los productos de este tipo de mayor difusi\'on son
\href{http://www.hibernate.org}{Hibernate},
\href{http://www.ibatis.com}{IBatis},
\href{http://db.apache.org/ojb}{Object Relational Bridge} y
\href{http://www,springframework.org}{Spring}.\\
Por el contrario, hay dise\~nos que necesitan sincronizarse con el
modelo relacional, ya que en su concepci\'on se parti\'o del mismo. En
esos casos, lo recomendable es utilizar un \textit{patr\'on de
dise\~no} denominado
\href{http://java.sun.com/blueprints/corej2eepatterns/Patterns/DataAccessObject.html}{\tech{Data
    Access Object} (\acronim{DAO})}, el
cual a\'i{}sla convenientemente los detalles del mecanismo de
persistencia respecto al dise\~no orientado a objetos de la
aplicaci\'on. \'Este es el enfoque utilizado por \queryj.

\section{Descripci\'on t\'ecnica}
\queryj es un producto software compuesto por los siguientes m\'odulos:
\begin{enumerate}
  \item Librer\'i{}a \tech{Java} de ejecuci\'on de operaciones
  \acronim{SQL},
  \item Herramienta de generaci\'on de c\'odigo a partir de un modelo
  relacional arbitrario:
  \begin{enumerate}
    \item API basado en patrones \acronim{DAO},
    \item Implementaci\'on \tech{JDBC} de dichos \acronim{DAO}s,
    \item Implementaciones alternativas (\acronim{XML}, \tech{Mock})
    \item Sincronizaci\'on de los metadatos del modelo como mecanismo
    de validaci\'on de consultas.
  \end{enumerate}
\end{enumerate}
\subsection{Librer\'i{}a \acronim{SQL}}
El objetivo de este m\'odulo es
proporcionar mecanismos de seguridad
en lo referente a la sintaxis y parametrizaci\'on de las operaciones
\acronim{SQL}.\\
Para ello, se dispone de un API que modela el lenguaje \acronim{SQL},
impidiendo expl\'i{}citamente la utilizaci\'on de sentencias no
v\'alidas. Al utilizarse conjuntamente con el resultado de la
generaci\'on de c\'odigo relativo a la sincronizaci\'on de metadatos,
permite comprobar y corregir las consultas \textbf{en tiempo de
  compilaci\'on}.
\subsection{Herramienta de generaci\'on de c\'odigo}
Mediante un an\'alisis exhaustivo de un modelo relacional dado, se
obtiene la informaci\'on necesaria para generar \textit{todo} el
c\'odigo de la capa de persistencia.\\
Este m\'odulo permite su integraci\'on con la herramienta de
desarrollo \tech{Java} m\'as difundida y de mayor aceptaci\'on,
\href{http://ant.apache.org}{Ant}. Esto permite, asimismo, una mayor
claridad y facilidad de configuraci\'on.\\
Para garantizar la robustez del c\'odigo generado, todas las \textbf{consultas
espec\'i{}ficas} definidas por el desarollador son \textbf{validadas
  en el proceso de generaci\'on}, con lo cual cualquier error en las mismas se
detecta en una fase temprana.
\subsubsection{Generaci\'on del API de la capa de persistencia basado en \acronim{DAO}}
El modelo generado se basa en la idea de establecer una granularidad
de un \acronim{DAO} por cada \textit{entidad} del modelo
relacional.\\
El API, adem\'as, genera los \tech{Value Objects} o \tech{Data
Transfer Objects} asociados, as\'i{}como un mecanismo flexible para
la selecci\'on de la implementaci\'on concreta del \acronim{DAO}.\\
Por conveniencia, para entidades \textit{d\'ebiles} del modelo,
utilizadas \'unicamente como identificadores de tipos, o tablas con
contenidos constantes en general, \queryj permite incluir
autom\'aticamente dichos en el API, de forma que la aplicaci\'on
pueda, si lo necesita, utilizar dichos valores sin realizar
operaciones \acronim{SQL} innecesarias.
\subsubsection{Generaci\'on de implementaciones \acronim{DAO}}
Cuando la aplicaci\'on utiliza el API generado,
est\'a ejecutando transparentemente l\'ogica \acronim{JDBC}, dentro de
la implementaci\'on espec\i{}ifica para el motor de base de datos
sobre el cual se despleg\'o el modelo relacional.\\
Dicha implementaci\'on utiliza fundamentalmente el m\'odulo de
validaci\'on descrito anteriormente, junto con \tech{Spring}, por su
capacidad para identificar errores espec\i{}ficos de cada base de
datos.\\
Es importante indicar asimismo que dicha implementaci\'on gestiona
adecuadamente las claves for\'aneas (\tech{foreign keys}), siendo
capaz de conocer si la eliminaci\'on de un registro de una tabla
conlleva la eliminaci\'on \textit{en cascada} de registros
dependientes. o no.\\
Adem\'as de la implementaci\'on por defecto, \queryj permite utilizar,
si los requisitos lo aconsejan, \acronim{DAO}s basados en
\acronim{XML} y \tech{Mock}. Los primeros pueden ser \'utiles a la
hora de realizar operaciones tales como importar o exportar datos,
mientras que el segundo es conveniente en la ejecuci\'on de pruebas
unitarias.\\
Cabe indicar que para todas las implementaciones se generan casos de
prueba basados en \href{http://www.junit.org}{JUnit}.
\subsubsection{Sincronizaci\'on del modelo}
En el caso de necesitar realizar operaciones \acronim{SQL}
independientemente de la capa de persistencia, \queryj proporciona un
API en \tech{Java} asociado al modelo analizado, incluyendo
\begin{enumerate}
  \item nombres de las tablas,
  \item nombres de las columnas,
  \item tipos de las columnas,
  \item extensiones \acronim{SQL} soportadas: funciones num\'ericas,
  sobre cadenas, sobre fechas, etc.
  \item procedimientos almacenados.
\end{enumerate}
Esto permite, dentro del \tech{IDE} de \tech{Java}, funciones como el
\textit{autocompletado} de nombres de tablas y columnas, validaci\'on
de tipos de columnas, consulta y utilizaci\'on de las extensiones
\acronim{SQL} soportadas, llamadas a procedimientos almacenados como
si se tratara de c\'odigo \tech{Java}, etc.
\section{Adaptabilidad}
La herramienta de generaci\'on de c\'odigo est\'a dise\~ntilde para
poder adaptar el resultado del proceso al estilo e incluso a la
tecnolog\'i{}a que se desee.
\section{Competencia}
\subsection{Firestorm}
\href{http://www.codefutures.com/products/firestorm/}{Firestorm/DAO}
\subsection{TopLink}
\href{http://www.oracle.com/technology/products/ias/toplink/}{TopLink}
\end{document}
